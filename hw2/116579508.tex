\documentclass[letterpaper,fleqn]{article}

\usepackage{enumerate}
\usepackage{amsmath}
\usepackage{amssymb}
\usepackage[letterpaper, margin=1in]{geometry}
\usepackage{tikz}

\begin{document}
	\begin{enumerate} 
		\item [] \textbf{Problem 1.}
		\begin{enumerate}[]
			\item Set 3.1
			\begin{enumerate}
				\item [10.]
				$\forall$ positive integers $m$ and $n$, $m \times n \geq m + n$ \\
				$m=0$, $n=1$
				$\begin{aligned}[t]
					0 \times 1 & \stackrel{?}{\geq} 0 + 1 \\
					0 & \not\geq 1
				\end{aligned}$
				
				\item [12.]
				$\forall$ real numbers $x$ and $y$, $\sqrt{x+y} = \sqrt{x}+\sqrt{y}$ \\
				$x=1$, $y=1$ \\
				$m=0$, $n=1$ \\
				$\begin{aligned}[t]
					\sqrt{1+1} & \stackrel{?}{=} \sqrt{1}+\sqrt{1} \\
					\sqrt{2} & \not= 2
				\end{aligned}$
				
			\end{enumerate}
		\end{enumerate}
		
		\item [] \textbf{Problem 2.}
		\begin{enumerate}[]
			\item Set 3.1
			\begin{enumerate}
				\item [29.] \quad
				\begin{enumerate}[(a)]
					\item
					$\exists x$ such that Rect($x$) $\wedge$ Square($x$) \\
					There are geometric figures that are both rectangles and squares. \\
					True; squares that are both rectangles and squares.
					\item
					$\exists x$ such that Rect($x$) $\wedge$ $\neg$Square($x$) \\ 
					There are geometric figures that are rectangles but not squares. \\
					True; rectangles of unequal side lengths are rectangles but not squares.
					\item
					$\forall x$, Square($x$) $\rightarrow$ Rect($x$) \\
					If a geometric figure is a square, it is a rectangle. \\
					True; squares have all the criteria of rectangles but have the added criteria of equal side lengths.
				\end{enumerate}
			\end{enumerate}
		\end{enumerate}
		
		\item [] \textbf{Problem 3.}
		\begin{enumerate}[]
			\item Set 3.1
			\begin{enumerate}
				\item [33.] \quad
				\begin{enumerate}
					\item [(c)]
					$ab = 0 \Rightarrow a = 0$ or $b=0$ \\
					True
					
					\item [(d)]
					$a < b$ and $c < d \Rightarrow ac < bd$ \\
					$a=-1$, $b=0$, $c=-1$, $d=0$ \\
					$\begin{aligned}[t]
						-1 \times -1 & \stackrel{?}{<} 0 \times 0 \\
						1 & \not < 0
					\end{aligned}$ \\
					False
				\end{enumerate}
			\end{enumerate}
		\end{enumerate}
		
		\item [] \textbf{Problem 4.}
		\begin{enumerate}[]
			\item Set 3.2
			\begin{enumerate}
				\item [10.]
				$\forall$ computer programs $P$, if $P$ compiles without error messages, then $P$ is correct. \\
				$\exists P$ such that $P$ compiles without error messages and isn't correct.
				
				\item [17.]
				$\forall$ integers $d$, if $6/d$ is an integer then $d=3$. \\
				$\exists d$ such that $6/d$ is an integer and $d \not= 3$.
				
				\item [19.]
				$\forall n \in \mathbb{Z}$, if $n$ is prime then $n$ is odd or $n=2$. \\
				$\exists n \in \mathbb{Z}$ such that $n$ is prime and $n$ is even and $n=2$.
				
				\item [21.]
				$\forall$ integers $n$, if $n$ is divisible by 6, then $n$ is divisible by 2 and $n$ is divisible by 3. \\
				$\exists n$ such that $n$ is divisible by 6 and not divisible by 2 and not divisible by 3. 
				
				\item [23.]
				If a function is differentiable then it is continuous. \\
				There exists a function that is differentiable and not continuous.
			\end{enumerate}
		\end{enumerate}
		
		\item [] \textbf{Problem 5.}
		\begin{enumerate}[]
			\item Set 3.2
			\begin{enumerate}
				\item [40.]
				Being divisible by 8 is a sufficient condition or being divisible by 4. \\
				If $n$ is divisible by 8, then $n$ is divisible by 4.
				
				\item [42.]
				Passing a comprehensive exam is a necessary condition for obtaining a master's degree. \\
				If one does not pass a comprehensive exam, then one cannot obtain a master's degree.
				
				\item [44.]
				Having a large income is not a necessary condition for a person to be happy. \\
				$\neg(\forall x (\text{Happy}(x) \rightarrow \text{HighIncome}(x)))$ \\
				$\exists x (\text{Happy}(x) \wedge \neg \text{HighIncome}(x)))$ \\
				There exists a happy person that doesn't have a large income.
				
				\item [46.]
				Being a polynomial is not a sufficient condition for a function to have a real root. \\
				$\neg(\forall x (\text{Polynomial}(x) \rightarrow \text{RealRoot}(x)))$ \\
				$\exists x (\text{Polynomial}(x) \wedge \neg\text{RealRoot}(x)))$ \\
				There exists a polynomial function without a real root.
				
				\item [47.]
				The absence of error messages during translation of a computer program is only a necessary and not a sufficient condition for reasonable [program] correctness. \\
				$\forall x (\text{Correct}(x) \rightarrow \neg\text{Error}(x) \wedge \neg\text{Error}(x) \not\rightarrow \text{Correct}(x))$ \\
				There are no error messages whenever a program is correct, but there are incorrect programs without error messages.
				
			\end{enumerate}
		\end{enumerate}
		
		\item [] \textbf{Problem 6.}
		\begin{enumerate}[]
			\item Set 3.3
			\begin{enumerate}
				\item [41.] \quad
				\begin{enumerate}
					\item [(c)]
					$\exists x \in \mathbb{R}$ such that $\forall y \in \mathbb{R}$, $x=y+1$ \\
					True because $\mathbb{R}$ is closed under addition.
					
					\item [(d)]
					$\begin{aligned}[t]
						\forall x \in \mathbb{R}^+ \text{, } \exists y \in \mathbb{R}^+ \text{ such that } xy &=1 \\
						y &=\frac{1}{x}
					\end{aligned}$ \\\\
					True because $\mathbb{R}^+$ is closed under division.
					
					\item [(f)]
					$\forall x \in \mathbb{Z}^+$ and $\forall y \in \mathbb{Z}^+$, $\exists z \in \mathbb{Z}^+$ such that $z=x-y$ \\
					False because $\mathbb{Z}^+$ isn't closed under subtraction. \\
					$x=1$, $y=2$ \\
					$z=1-2=-1 \therefore z \not\in \mathbb{Z}^+$.
					
					\item [(g)]
					$\forall x \in \mathbb{Z}$ and $\forall y \in \mathbb{Z}$, $\exists z \in \mathbb{Z}$ such that $z=x-y$ \\
					True because $\mathbb{Z}$ is closed under subtraction.
					
					\item [(h)]
					$\begin{aligned}[t]
						\exists u \in \mathbb{R}^+ \text{ such that } \forall v \in \mathbb{R}^+ \text{, } uv &<v \\
						u &< 1
					\end{aligned}$ \\\\
					True because $\{u \in \mathbb{R}^+ \,|\, u<1\} \subset \mathbb{R}^+$.
					
				\end{enumerate}
			\end{enumerate}
		\end{enumerate}
		
		\item [] \textbf{Problem 7.}
		\begin{enumerate}[]
			\item Set 3.4
			\begin{enumerate}
				\item [13.]
				$\begin{aligned}[t]
					& \text{For all students $x$, if $x$ studies discrete mathematics, then
					$x$ is good at logic.} \\
					& \text{Tarik studies discrete mathematics.} \\
					\therefore{} & \text{Tarik is good at logic.}
				\end{aligned}$ \\\\
				Valid by universal modus ponens.
				
				\item [14.]
				$\begin{aligned}[t]
					& \text{If compilation of a computer program produces error messages, then the program is not correct.} \\
					& \text{Compilation of this program does not produce error messages.} \\
					\therefore{} & \text{This program is correct.}
				\end{aligned}$ \\\\
				Invalid by inverse error.
				
				\item [15.]
				$\begin{aligned}[t]
					& \text{Any sum of two rational numbers is rational.} \\
					& \text{The sum $r+s$ is rational.} \\
					\therefore{} & \text{The numbers $r$ and $s$ are both rational.}
				\end{aligned}$ \\\\
				Invalid by converse error.
				
				\item [17.]
				$\begin{aligned}[t]
					& \text{If an infinite series converges, then the terms go to 0.} \\
					& \text{The terms of the infinite series $\sum_{n=1}^{\infty}\frac{1}{n}$ go to 0.} \\
					\therefore{} & \text{The infinite series $\sum_{n=1}^{\infty}\frac{1}{n}$ converges.} 
				\end{aligned}$ \\\\
				Invalid by converse error.
				
				\item [18.]
				$\begin{aligned}[t]
					& \text{If an infinite series converges, then the terms go to 0.} \\
					& \text{The terms of the infinite series $\sum_{n=1}^{\infty}\frac{1}{n+1}$ do not go to 0.} \\
					\therefore{} & \text{The infinite series $\sum_{n=1}^{\infty}\frac{1}{n+1}$ does not converge.} 
				\end{aligned}$ \\\\
				Valid by universal modus tollens.
				
			\end{enumerate}
		\end{enumerate}
		
		\item [] \textbf{Problem 8.}
		\begin{enumerate}[]
			\item Set 3.4
			\begin{enumerate}
				\newcommand{\diagram}[1]{
				\begin{tikzpicture}[baseline={(current bounding box.north)}, font=\scriptsize]
				\draw[align=center]
				#1
				;
				\end{tikzpicture}}
				\newcommand{\bubble}[4]{
				(#1) circle (#2) node[black, yshift=#3] {#4}
				}
				\newcommand{\obubble}[1]{\bubble{0,0}{1.5}{0.4in}{#1}}
				\newcommand{\ibubble}[1]{\bubble{0,-0.4}{1}{0}{#1}}
				\newcommand{\dlabel}[2]{(#1) node {#2}}
				\newcommand{\ddlabel}[1]{\dlabel{0,-2}{#1}}
				
				\item [22.]
				$\begin{aligned}[t]
					& \text{All discrete mathematics students can tell a valid argument from an invalid one.} \\
					& \text{All thoughtful people can tell a valid argument from an invalid one.} \\
					\therefore{} & \text{All discrete mathematics students are thoughtful.} 
				\end{aligned}$ \\\\
				Invalid. \\
				\diagram{
				\obubble{Validity tellers}
				\ibubble{Discrete math\\students (S)}
				\ddlabel{Major premise}
				}
				\diagram{
				\obubble{Validity tellers}
				\ibubble{Thoughtful\\people (P)}
				\ddlabel{Minor premise}
				} \\
				
				\diagram{
				\bubble{0,0}{1.5}{0.4in}{Validity tellers}
				\bubble{-0.7,-0.4}{0.6}{0in}{S}
				\bubble{0.7,-0.4}{0.6}{0in}{P}
				\ddlabel{(a)}
				}
				\diagram{
				\bubble{0,0}{1.5}{0.4in}{Validity tellers}
				\bubble{-0.4,-0.4}{0.6}{0in}{S}
				\bubble{0.4,-0.4}{0.6}{0in}{P}
				\ddlabel{(b)}
				}
				\diagram{
				\bubble{0,0}{1.5}{0.4in}{Validity tellers}
				\bubble{0,-0.4}{1}{0in}{S=P}
				\ddlabel{(c)}
				} \\
				\newpage
				
				\item [23.]
				$\begin{aligned}[t]
					& \text{All teachers occasionally make mistakes.} \\
					& \text{No gods ever make mistakes.} \\
					\therefore{} & \text{No teachers are gods.} 
				\end{aligned}$ \\\\
				Valid.\\
				\diagram{
				\obubble{Mistake makers}
				\ibubble{Teachers}
				\ddlabel{Major premise}
				}
				\diagram{
				\bubble{0,0}{1.5}{0}{Gods}
				\ddlabel{Minor premise}
				} \\
				\diagram{
				\obubble{Mistake makers}
				\ibubble{Teachers}
				\bubble{3.25,0}{1.5}{0}{Gods}
				\dlabel{1.625,-2}{(a)}
				}
				
				\item [24.]
				$\begin{aligned}[t]
					& \text{No vegetarians eat meat.} \\
					& \text{All vegans are vegetarian.} \\
					\therefore{} & \text{No vegans eat meat.} 
				\end{aligned}$ \\\\
				Valid. \\
				\diagram{
				\obubble{Doesn't eat\\meat}
				\ibubble{Vegetarians}
				\ddlabel{Major premise}
				}
				\diagram{
				\obubble{Vegetarians}
				\ibubble{Vegans}
				\ddlabel{Minor premise}
				} \\
				\diagram{
				\bubble{0,0}{2}{0.6in}{Doesn't eat\\meat}
				\bubble{0,-0.4}{1.5}{0.4in}{Vegetarians}
				\bubble{0,-0.8}{1}{0}{Vegans}
				\dlabel{0,-2.5}{(a)}
				}
				\newpage
				
				\item [26.]
				$\begin{aligned}[t]
					& \text{All polynomial functions are differentiable.} \\
					& \text{All differentiable functions are continuous.} \\
					\therefore{} & \text{All polynomial functions are continuous.} 
				\end{aligned}$ \\\\
				Valid. \\
				\diagram{
				\obubble{Differentiable}
				\ibubble{Polynomial}
				\ddlabel{Major premise}
				}
				\diagram{
				\obubble{Continuous}
				\ibubble{Differentiable}
				\ddlabel{Minor premise}
				} \\
				\diagram{
				\bubble{0,0}{2}{0.6in}{Continuous}
				\bubble{0,-0.4}{1.5}{0.4in}{Differentiable}
				\bubble{0,-0.8}{1}{0}{Polynomial}
				\dlabel{0,-2.5}{(a)}
				}
				
				\item [27.]
				$\begin{aligned}[t]
					& \text{Nothing intelligible ever puzzles me.} \\
					& \text{Logic puzzles me.} \\
					\therefore{} & \text{Logic is unintelligible.} 
				\end{aligned}$ \\\\
				Valid. \\
				\diagram{
				\bubble{-1.625,0}{1.5}{0}{Intelligible}
				\bubble{1.625,0}{1.5}{0}{Puzzling}
				\ddlabel{Major Premise}
				} 
				\diagram{
				\obubble{Puzzling}
				\ibubble{Logic}
				\ddlabel{Minor Premise}
				} \\
				\diagram{
				\bubble{-1.625,0}{1.5}{0}{Intelligible}
				\bubble{1.625,0}{1.5}{0.4in}{Puzzling}
				\bubble{1.625,-0.4}{1}{0}{Logic}
				\ddlabel{(a)}
				}
			\end{enumerate}
		\end{enumerate}
	\end{enumerate}
\end{document}