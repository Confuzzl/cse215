\documentclass[letterpaper,fleqn,leqno]{article}

\usepackage{amsmath}
\usepackage{amssymb}
\usepackage{enumerate}
\usepackage[letterpaper, margin=1in]{geometry}
%\usepackage{xcolor, colortbl}
%\usepackage{cancel}
%\usepackage{tabularx}


\begin{document}
	\newcommand{\chunk}[1]{
		\item 
		\begin{minipage}[t]{\textwidth}
			#1
		\end{minipage}
	}
	\newcommand{\set}[3]{
		\chunk{
			Exercise Set 5.#1, Problem #2 \\
			#3
		}
	}
	\newcommand{\sets}[3]{
		\chunk{
			Exercise Set 5.#1, Problems #2 \\
			#3
		}
	}
%	\newcommand{\set}[2]{Exercise Set 5.#1, Problem #2}
%	\newcommand{\sets}[2]{Exercise Set 5.#1, Problems #2}
	\newcommand{\Sum}[2]{\displaystyle\sum\limits_{#1}^{#2}}
	\newcommand{\Prod}[2]{\displaystyle\prod\limits_{#1}^{#2}}
	
	\noindent \textbf{Problem 1.}
	\begin{enumerate}[(a)]
		\set{2}{11}{
			$1^3+2^3+\dots+n^3=\left[\dfrac{n(n+1)}{2}\right]^2$, for all integers $n\geq1$. \\
			Let property $P(n)$ be $\Sum{i=1}{n} i^3=\left[\dfrac{n(n+1)}{2}\right]^2$. \\
			\textbf{Basis:} \\
			$P(1): \left(\Sum{i=1}{1} i^3=1\right)=\left(\left[\dfrac{n(n+1)}{2}\right]^2=1\right)$ is true. \\
			\textbf{Inductive hypothesis:} \\
			Assume $P(k): \Sum{i=1}{k} i^3=\left[\dfrac{k(k+1)}{2}\right]^2$ for $k\geq1\in\mathbb{Z}$ is true. \\
			$P(k+1):$ \\
			$\begin{aligned}[t]
				\Sum{i=1}{k+1} i^3 &= \left[\dfrac{(k+1)(k+2)}{2}\right]^2 \\
				\left(\Sum{i=1}{k}i^3\right)+(k+1)^3 &= \left[\dfrac{k^2+3k+2}{2}\right]^2 \\
				\left[\dfrac{k(k+1)}{2}\right]^2+k^3+3k^2+3k+1 &= \left[\dfrac{k^2+3k+2}{2}\right]^2 \\
				\left[\dfrac{k^2+k}{2}\right]^2+k^3+3k^2+3k+1 &= \left[\dfrac{k^2+3k+2}{2}\right]^2 \\
				\dfrac{k^4+2k^3+k^2}{4}+\dfrac{4k^3+12k^2+12k+4}{4} &= \dfrac{k^4+3k^3+2k^2+3k^3+9k^2+6k+2k^2+6k+4}{4} \\
				\dfrac{k^4+6k^3+13k^2+12k+4}{4} &= \dfrac{k^4+6k^3+13k^2+12k+4}{4} \\
			\end{aligned}$ \\
			Basis and inductive hypothesis proven, therefore original statement is true. \\
		}
			
		\set{2}{14}{
			$\Sum{i=1}{n+1}i\cdot2^i=n\cdot2^{n+2}+2$, for all integers $n\geq0$. \\
			Let property $P(n)$ be $\Sum{i=1}{n+1}i\cdot2^i=n\cdot2^{n+2}+2$. \\
			\textbf{Basis:} \\
			$P(0): \left(\Sum{i=1}{1}i\cdot2^i=2\right)=\left(0\cdot2^{2}+2=2\right)$ is true. \\
			\textbf{Inductive hypothesis:} \\
			Assume $P(k): \Sum{i=1}{k+1}i\cdot2^i=k\cdot2^{k+2}+2$ for $k\geq0\in\mathbb{Z}$ is true. \\
			$P(k+1):$ \\
			$\begin{aligned}[t]
				\Sum{i=1}{k+2}i\cdot2^i &= (k+1)\cdot2^{k+3}+2 \\
				\left(\Sum{i=1}{k+1}i\cdot2^i\right) + (k+2)\cdot2^{k+2} &= (k+1)\cdot2^{k+3}+2 \\
				k\cdot2^{k+2}+2+(k+2)\cdot2^{k+2} &= (k+1)\cdot2^{k+3}+2 \\
				k\cdot2^{k+2}+2+k\cdot2^{k+2}+2^{k+3} &= (k+1)\cdot2^{k+3}+2 \\
				2k\cdot2^{k+2}+2+2^{k+3} &= (k+1)\cdot2^{k+3}+2 \\
				k\cdot2^{k+3}+2+2^{k+3} &= (k+1)\cdot2^{k+3}+2 \\
				(k+1)\cdot2^{k+3}+2 &= (k+1)\cdot2^{k+3}+2 \\
			\end{aligned}$ \\
			Basis and inductive hypothesis proven, therefore original statement is true. \\
		}
		
		\set{2}{17}{
			$\Prod{i=0}{n}\left(\dfrac{1}{2i+1}\cdot\dfrac{1}{2i+2}\right)=\dfrac{1}{(2n+2)!}$, for all integers $n\geq0$. \\
			Let property $P(n)$ be $\Prod{i=0}{n}\left(\dfrac{1}{2i+1}\cdot\dfrac{1}{2i+2}\right)$. \\
			\textbf{Basis:} \\
			$P(0): \left(\Prod{i=0}{0}\left(\dfrac{1}{2i+1}\cdot\dfrac{1}{2i+2}\right)=\dfrac{1}{2}\right)=\left(\dfrac{1}{(0+2)!}=\dfrac{1}{2}\right)$ is true. \\
			\textbf{Inductive Hypothesis:} \\
			Assume $P(k): \Prod{i=0}{k}\left(\dfrac{1}{2i+1}\cdot\dfrac{1}{2i+2}\right)=\dfrac{1}{(2k+2)!}$ for $k\geq0\in\mathbb{Z}$ is true. \\
			$P(k+1):$ \\
			$\begin{aligned}[t]
				\Prod{i=0}{k+1}\left(\dfrac{1}{2i+1}\cdot\dfrac{1}{2i+2}\right) &= \dfrac{1}{(2(k+1)+2)!} \\
				\Prod{i=0}{k}\left(\dfrac{1}{2i+1}\cdot\dfrac{1}{2i+2}\right)\left(\dfrac{1}{2(k+1)+1}\cdot\dfrac{1}{2(2k+1)+2}\right) &= 
				\dfrac{1}{(2k+4)!} \\
				\dfrac{1}{(2k+2)!}\left(\dfrac{1}{2k+3}\cdot\dfrac{1}{2k+4}\right) &= \dfrac{1}{(2k+4)!} \\
				\dfrac{1}{(2k+2)!(2k+3)(2k+4)} &= \dfrac{1}{(2k+4)(2k+3)!} \\
				\dfrac{1}{(2k+4)(2k+3)(2k+2)!} &= \dfrac{1}{(2k+4)(2k+3)(2k+2)!} \\
			\end{aligned}$ \\
			Basis and inductive hypothesis proven, therefore original statement is true. \\
		}
		
		\set{3}{10}{
			$n^3-7n+3$ is divisible by 3, for each integer $n\geq0$. \\
			Let property $P(n)$ be $n^3-7n+3$ is divisible by 3. \\
			\textbf{Basis:} \\
			$P(0): (0^3-7\cdot0+3=0)$ is divisible by 3 is true. \\
			\textbf{Inductive Hypothesis:} \\
			Assume $P(k): k^3-7k+3$ is divisible by 3 where $k\geq0\in\mathbb{Z}$ is true. \\
			
			$\begin{aligned}[t]
				P(k+1):{} & (k+1)^3-7(k+1)+3 \\
				& k^3+3k^2+3k+1-7k-7+3 \\
				& \underbrace{\left(k^3-7k+3\right)}_{3a}+3k^2+3k+1-7 \\
				& \text{$P(k)$ true, therefore multiple of some integer $a$} \\
				& 3a+3k^2+3k-6 \\
				& 3a+3k^2+3k-6 \\
				& 3\left(a+k^2+k-2\right) \\
			\end{aligned}$ \\
			Expression is a multiple of 3, therefore $P(k+1)$ is divisible by 3. \\
			Basis and inductive hypothesis proven, therefore original statement is true. \\
		}
		
		\set{3}{17}{
			$1+3n\leq4^n$, for every integer $n\geq0$. \\
			Let property $P(n)$ be $1+3n\leq4^n$. \\
			\textbf{Basis:} \\
			$P(0): (1+3\cdot0=1)\leq(4^0=1)$ is true. \\
			\textbf{Inductive hypothesis:} \\
			Assume $P(k): 1+3k\leq4^k$ for $k\geq0\in\mathbb{Z}$ is true. \\
			$P(k+1):$ \\
			$\begin{aligned}[t]
				1+3(k+1) &\leq 4^{k+1} \\
				4+3k &\leq 4^{k+1} \\
				1+3k+3 &\leq 4^{k+1} \\
				P(k)+3 &\leq 4^{k+1} \\
				4^k+3 &\leq 4^{k+1} \\
				3 &\leq 4^{k+1}-4^k \\
				3 &\leq 4^k(4-1) \\
				3 &\leq 3\cdot4^k \\
				1 &\leq 4^k \\
			\end{aligned}$ \\
			Last inequality holds true for all $k\geq0\in\mathbb{Z}$. \\
			Basis and inductive hypothesis proven, therefore original statement is true. \\
		}
		
		
		\set{3}{21}{
			$\sqrt{n}<\dfrac{1}{\sqrt{1}}+\dfrac{1}{\sqrt{2}}+\dots+\dfrac{1}{\sqrt{n}}$, for all integers $n\geq2$. \\
			Let property $P(n)$ be $\sqrt{n}<\Sum{i=1}{n}\dfrac{1}{\sqrt{i}}$. \\
			\textbf{Basis:} \\
			$P(2):(\sqrt{2})<\left(\Sum{i=1}{2}\dfrac{1}{\sqrt{i}}=\dfrac{1}{\sqrt{1}}+\dfrac{1}{\sqrt{2}}=1+\dfrac{\sqrt{2}}{2}\right)$ is true. \\
			\textbf{Inductive hypothesis:} \\
			Assume $P(k):\sqrt{k}<\Sum{i=1}{k}\dfrac{1}{\sqrt{i}}$ for $k\geq2\in\mathbb{Z}$ is true. \\
			$P(k+1):$ \\
			$\begin{aligned}[t]
				\sqrt{k+1} &< \Sum{i=1}{k+1}\dfrac{1}{\sqrt{i}} \\
				\sqrt{k+1} &< \Sum{i=1}{k}\dfrac{1}{\sqrt{i}}+\dfrac{1}{\sqrt{k+1}} \\
				\sqrt{k+1} &< \sqrt{k}+\dfrac{1}{\sqrt{k+1}} \\
				\dfrac{k+1}{\sqrt{k+1}} &< \dfrac{\sqrt{k}\sqrt{k+1}+1}{\sqrt{k+1}} \\
				k+1 &< \sqrt{k}\sqrt{k+1}+1 \\
				\sqrt{k}\sqrt{k} &< \sqrt{k}\sqrt{k+1} \\
				k &< k+1 \\
			\end{aligned}$ \\
			Inequality holds true for all $k\geq2$. \\
			Basis and inductive hypothesis proven, therefore original statement is true. \\
		}
		
		
		
		\set{3}{22}{
			$1+nx\leq(1+x)^n$, for all real numbers $x>-1$ and integers $n\geq2$. \\
			Let property $P(n,x)$ be $1+nx\leq(1+x)^n$. \\
			\textbf{Basis:} \\
			$P(2,x):1+2x\leq(1+x)^2\Rightarrow1+2x\leq1+2x+x^2\Rightarrow0\leq x^2$ is true for all $x\in\mathbb{R}$. \\
			\textbf{Inductive hypothesis:} \\
			Assume $P(k,x):1+kx\leq(1+x)^k$ for $k\geq2\in\mathbb{Z}$ and $x>-1\in\mathbb{R}$ is true. \\
			$P(k+1,x):$ \\
			$\begin{aligned}[t]
				1+(k+1)x &\leq (1+x)^{k+1} \\
				1+kx+x &\leq (1+x)^{k+1} \\
				(1+x)^k+x &\leq (1+x)^{k+1} \\
				x &\leq (1+x)^k((1+x)-1) \\
				x &\leq (1+x)^kx \\
			\end{aligned}$ \\
			\begin{tabular}{|l|}
				\hline
				Case 1: $x=0$ \\
				\hline
				Prove: $0 \leq (1+0)^k\cdot0$ \\
				$0 \leq (1+0)^k\cdot0$ \\
				$0 \leq 0$ \\
				Proof done. \\
				\hline
				Case 2: $x>0$ \\
				\hline
				Prove: $x \leq (1+x)^kx$ \\
				\\
				\begin{tabular}{l|l}
					$\begin{aligned}[t]
						& x \leq (1+x)^kx \\
						={} & 1 \leq (1+x)^k \\ 
					\end{aligned}$ &
					$\begin{aligned}[t]
						x &> 0 \\
						1 &< 1+x \\
						1^k &< (1+x)^k \\
						1 &< (1+x)^k \\
					\end{aligned}$ \\
				\end{tabular} \\
				\\
				$1<(1+x)^k \Rightarrow 1\leq(1+x)^k$ \\
				Proof done. \\
				\hline
			\end{tabular}
			\begin{tabular}{|l|}
				\hline
				Case 3: $-1<x<0$ \\
				\hline
				Prove: $x \leq (1+x)^kx$ \\
				\\
				\begin{tabular}{l|l}
					$\begin{aligned}[t]
						& x \leq (1+x)^kx \\
						={} & 1 \geq (1+x)^k \\ 
					\end{aligned}$ &
					$\begin{aligned}[t]
						0 &>& x &>-1 \\
						1 &>& 1+x &>0 \\
						1^k &>& (1+x)^k &>0^k \\
						1 &>& (1+x)^k &>0 \\
					\end{aligned}$ \\
				\end{tabular} \\
				\\
				$1>(1+x)^k>0 \Rightarrow 1\geq(1+x)^k$ \\
				Proof done. \\
				\hline
			\end{tabular} \\
			$P(k+1,x)$ for $k\geq2\in\mathbb{Z}$ and $x>-1\in\mathbb{R}$ is true in all cases. \\
			Basis and inductive hypothesis proven, therefore original statement is true. \\
		}
		\set{3}{29}{
			As each of a group of businesspeople arrives at a meeting, each shakes hands with all the other people present. Use mathematical induction to show that if $n$ people come to the meeting then $[n(n-1)]/2$ handshakes occur. \\
			If the set of businesspeople has size $n$, then the number of handshakes is $\displaystyle \binom{n}{2}$. \\
			For a set of 0 and 1 businesspeople, no handshakes occur. \\
			$\displaystyle \binom{n}{2}=\dfrac{n!}{2!(n-2)!}=\dfrac{n!}{2(n-2)!}$ \\
			Let property $P(n)$ be $\begin{cases}
				0=\dfrac{n(n-1)}{2} & n=0,\, n=1 \\
				\dfrac{n!}{2(n-2)!}=\dfrac{n(n-1)}{2} & n\geq2 \\
			\end{cases}$ for non negative integers $n$. \\
			\textbf{Basis:} \\
			$P(0):0=\left(\dfrac{0(0-1)}{2}=0\right)$ is true. \\
			$P(1):0=\left(\dfrac{1(1-1)}{2}=0\right)$ is true. \\
			$P(2): \left(\dfrac{2!}{2(2-2)!}=\dfrac{2}{2\cdot1}=1\right)=\left(\dfrac{2(2-1)}{2}=1\right)$ is true. \\
			\textbf{Inductive hypothesis:} \\
			Assume $P(k): \dfrac{k!}{2(k-2)!}=\dfrac{k(k-1)}{2}$ for $k\geq0\in\mathbb{Z}$ to be true. \\
			$P(k+1):$ \\
			$\begin{aligned}[t]
				\dfrac{(k+1)!}{2((k+1)-2)!} &= \dfrac{(k+1)((k+1)-1)}{2} \\
				\dfrac{(k+1)k!}{2(k-1)!} &= \dfrac{(k+1)k}{2} \\
				\dfrac{(k+1)k(k-1)!}{2(k-1)!} &= \dfrac{(k+1)k}{2} \\
				\dfrac{(k+1)k}{2} &= \dfrac{(k+1)k}{2} \\
			\end{aligned}$ \\
			Basis and inductive hypothesis proven, therefore original statement is true. \\
		}
		\chunk{
			Prove that in an $n$-sided regular polygon, where $n\geq3$, the number of diagonals is $n(n-3)/2$. \\
			The number of possible vertex pairs in an $n$-sided regular polygon is $\displaystyle \binom{n}{2}$, and $n$ of these vertex pairs are the edges of the polygon. The number of diagonals is $\displaystyle \binom{n}{2}-n$. \\
			$\displaystyle \binom{n}{2}-n=\dfrac{n!}{2!(n-2)!}-n=\dfrac{n!}{2(n-2)!}-n$ \\
			Let property $P(n)$ be $\dfrac{n!}{2(n-2)!}-n=\dfrac{n(n-3)}{2}$. \\
			\textbf{Basis:} \\
			$P(3): \left(\dfrac{3!}{2(3-2)!}-3=\dfrac{6}{2}-3=0\right)=\left(\dfrac{3(3-3)}{2}=0\right)$ is true. \\
			\textbf{Inductive hypothesis:} \\
			Assume $P(k): \dfrac{k!}{2(k-2)!}-k=\dfrac{k(k-3)}{2}$ for $k\geq3$ to be true. \\
			$P(k+1):$ \\
			$\begin{aligned}[t]
				\dfrac{(k+1)!}{2((k+1)-2)!}-(k+1) &= \dfrac{(k+1)((k+1)-3)}{2} \\
				\dfrac{(k+1)k!}{2(k-1)!}-k-1 &= \dfrac{(k+1)(k-2)}{2} \\
				\dfrac{(k+1)k(k-1)!}{2(k-1)!}-k-1 &= \dfrac{(k+1)(k-2)}{2} \\
				\dfrac{(k+1)k}{2}-\dfrac{2k+2}{2} &= \dfrac{(k+1)(k-2)}{2} \\
				\dfrac{k^2+k-2k-2}{2} &= \dfrac{k^2-k-2}{2} \\
				\dfrac{k^2-k-2}{2} &= \dfrac{k^2-k-2}{2} \\
			\end{aligned}$ \\
			Basis and inductive hypothesis proven, therefore original statement is true. \\
		}
		\chunk{
			Prove that the number of permutations of the set $\left\{1,2,\dots,n\right\}$ with $n$ elements is $n!$, for natural number $n\geq1$. \\
			A set of $n=1$ elements has $1!=1$ permutation. \\
			Let property $P(n)$ be $\left\{1,2,\dots,n\right\}$ has $n!$ permutations. \\
			\textbf{Basis:} \\
			$P(1): \left\{1\right\}$ has $1!=1$ permutation is true. \\
			\textbf{Inductive hypothesis:} \\
			Assume $P(k): \left\{1,2,\dots,k\right\}$ has $k!$ permutations for $k\geq1$ to be true. \\
			In order to create a permuted set $B_p$ of size $k+1$, one can insert $k+1$ into $A_p$, an arbitrary permutation of set $A$ of $k$ elements. \\
			This action is equivalent to permuting a set of size $k+1$. \\
			There are $k+1$ positions to insert such an element into $A_p$: $k$ positions before each element and one position after the last element of $A_p$. \\
			There are $k!$ possible $A_p$ made from $A$. \\
			($k+1$ ways to insert into $A_p$) $\times$ ($k!$ possible $A_p$) = $(k+1)k! = (k+1)!$ ways to create $B_p$. \\
			$P(k+1)$ is true. \\
			Basis and inductive hypothesis proven, therefore original statement is true. \\
		} 
	\end{enumerate}
	
	\noindent \textbf{Problem 2.}
	\begin{enumerate}[(a)]
		\set{4}{8}{
			Suppose that $h_0,h_1,h_2,\dots$ is a sequence defined as follows: \\
			$h_0=1$, $h_1=2$, $h_2=3$, \\
			$h_k=h_{k-1}+h_{k-2}+h_{k-3}$ for all integers $k\geq3$. \\
			\begin{enumerate}[(a)]
				\item
				Prove that $h_n\leq3^n$ for all integers $n\geq0$. \\ 
				Let property $P(n)$ be $h_n\leq3^n$ for all integers $n\geq0$. \\
				\textbf{Basis:} \\
				$P(0): (h_0=1)\leq(3^0=1)$ is true. \\
				$P(1): (h=1=2)\leq(3^1=3)$ is true. \\
				$P(2): (h_2=3)\leq(3^2=9)$ is true. \\
				\textbf{Inductive hypothesis:} \\
				Let $k\geq2$. \\
				Assume $P(i): h_i=h_{i-1}+h_{i-2}+h_{i-3}\leq3^i$ for $0\leq i\leq k$ and $i\in\mathbb{Z}$. \\
				$P(k+1):$ \\
				$h_{k+1}\begin{aligned}[t]
					= & h_k+h_{k-1}+h_{k-2} \\
					\leq & 3^{k}+3^{k-1}+3^{k-2} \\
					\leq & 3^{k-2}\left(3^2+3+1\right) \\
					\leq & 13\cdot3^{k-2} \leq \left(3^3\cdot3^{k-2}=3^{k+1}\right) \\
				\end{aligned}$ \\
				$P(k+1): h_{k+1}\leq3^{k+1}$ is true. \\
				Basis and inductive hypothesis proven, therefore original statement is true. \\
				
				\item
				Suppose that $s$ is any real number such that $s^3\geq s^2+s+1$. (This implies that $2>s>1.83$.) Prove that $h_n\leq s^n$ for all $n\geq2$. \\
				Let property $P(n)$ be $h_n\leq s^n$ for $n\geq2\in\mathbb{Z}$. \\
				\textbf{Basis:} \\
				$P(2): (h_2=2)\leq(3.34<s^2<4)$ is true. \\
				$P(3): (h_3=6)\leq(6.12<s^3<8)$ is true. \\
				$P(4): (h_4=11)\leq(11.21<s^4<16)$ is true. \\
				\textbf{Inductive hypothesis:} \\
				Let $k\geq4$. \\
				Assume $P(i): h_i=h_{i-1}+h_{i-2}+h_{i-3}\leq s^i$ for $2\leq i\leq k$ and $i\in\mathbb{Z}$. \\
				$P(k+1):$ \\
				$h_{k+1}\begin{aligned}[t]
					= & h_k+h_{k-1}+h_{k-2} \\
					\leq & s^k+s^{k-1}+s^{k-2} \\
					\leq & s^{k-2}\left(s^2+s+1\right) \\
					\leq & s^{k-2}s^3 \\
					\leq & s^{k+1} \\
				\end{aligned}$ \\
				$P(k+1):h_{k+1}\leq s^{k+1}$ is true. \\
				Basis and inductive hypothesis proven, therefore original statement is true. \\
			\end{enumerate}
		}
		\set{4}{9}{
			Define a sequence $a_1,a_2,a_3,\dots$ as follows: $a_1=1$, $a_2=3$, and $a_k=a_{k-1}+a_{k-2}$ for all integers $k\geq3$. Use strong mathematical induction to prove that $a_n\leq\left(\dfrac{7}{4}\right)^n$ for all integers $n\geq1$. \\
			Let property $P(n)$ be $a_n\leq\left(\dfrac{7}{4}\right)^n$. \\
			\textbf{Basis:} \\
			$P(1): 1\leq\dfrac{7}{4}$ is true. \\
			$P(2): 3\leq\dfrac{49}{16}$ is true. \\
			\textbf{Inductive hypothesis:} \\
			Let $k\geq2$. \\
			Assume $P(i): a_i=a_{i-1}+a_{i-2}\leq\left(\dfrac{7}{4}\right)^i$ for $1\leq i\leq k$ and $i\in\mathbb{Z}$. \\
			$P(k+1):$ \\
			$a_{k+1}\begin{aligned}[t]
				=  & a_k+a_{k-1} \\
				\leq & \left(\dfrac{7}{4}\right)^k+\left(\dfrac{7}{4}\right)^{k-1} \\
				\leq & \left(\dfrac{7}{4}\right)^{k-1}\left(1+\dfrac{7}{4}\right) \\
				\leq & \left(\dfrac{11}{4}\left(\dfrac{7}{4}\right)^{k-1}=\dfrac{44}{16}\left(\dfrac{7}{4}\right)^{k-1}\right)\leq\left(\dfrac{49}{16}\left(\dfrac{7}{4}\right)^{k-1}=\left(\dfrac{7}{4}\right)^2\left(\dfrac{7}{4}\right)^{k-1}=\left(\dfrac{7}{4}\right)^{k+1}\right) \\
			\end{aligned}$ \\
			$P(k+1):a_{k+1}\leq\left(\dfrac{7}{4}\right)^{k+1}$ is true. \\
			Basis and inductive hypothesis proven, therefore original statement is true. \\
		}
		\set{4}{25(b)}{}
		\set{4}{30}{}
		\set{5}{30}{}
		\chunk{
			Let $f$ be a function on whole numbers satisfying \\
			$f(n)=
			\begin{cases}
				0 & \text{if $n=0$} \\	
				4f(n/2) & \text{if $n>0$ and even} \\	
				f(n-1)+2n-1 & \text{if $n>0$ and odd} \\	
			\end{cases}$ \\
			Prove that $f(n)=n^2$ for all $n\geq0$ \\
		}
	\end{enumerate}
	
	\noindent \textbf{Problem 3.}
	\begin{enumerate}[(a)]
		\sets{5}{28, 29}{}
		\set{6}{2(b,d)}{}
		\sets{6}{9, 14, 15}{}
	\end{enumerate}
\end{document}